\section*{АННОТАЦИЯ}

Техническое задание – это основной документ, оговаривающий набор требований и
порядок создания программного продукта, в соответствии с которым производится разработка программы, ее тестирование и приемка.

Настоящее Техническое задание на разработку ``3D Xоррор-игра от первого лица на Unreal Engine 5 для ПК'' содержит следующие разделы: «Введение», «Основание для разработки», «Назначение разработки», «Требования к программе», «Требования к программным документам», «Технико-экономические показатели», «Стадии и этапы разработки», «Порядок контроля и приемки».

В разделе «Введение» указано наименование и краткая характеристика области применения ``3D Xоррор-игра от первого лица на Unreal Engine 5 для ПК''

В разделе «Основания для разработки» указан документ, на основании которого ведется разработка, и наименование темы разработки.

В разделе «Назначение разработки» указано функциональное и эксплуатационное
назначение программного продукта.

Раздел «Требования к программе» содержит основные требования к функциональным характеристикам, к надежности, к условиям эксплуатации, к составу и параметрам технических средств, к информационной и программной совместимости.

Раздел «Требования к программным документам» содержит предварительный состав программной документации и специальные требования к ней.

Раздел «Технико-экономические показатели» содержит ориентировочную экономическую эффективность, предполагаемую годовую потребность, экономические преимущества разработки 

Раздел «Стадии и этапы разработки» содержит стадии разработки, этапы и содержание работ.

В разделе «Порядок контроля и приемки» указаны общие требования к приемке работы.

Настоящий документ разработан в соответствии с требованиями:
\begin{enumerate}
    \item ГОСТ 19.101-77 Виды программ и программных документов [1]
    \item ГОСТ 19.102-77 Стадии разработки [2]
    \item ГОСТ 19.103-77 Обозначения программ и программных документов [3]
    \item ГОСТ 19.104-78 Основные надписи [4]
    \item ГОСТ 19.105-78 Общие требования к программным документам [5]
    \item ГОСТ 19.106-78 Требования к программным документам, выполненным печатным способом [6]
    \item ГОСТ 19.201-78 Техническое задание. Требования к содержанию и оформлению [7]
\end{enumerate}

Изменения к данному Техническому заданию оформляются согласно ГОСТ 19.603-78 [12], ГОСТ 19.604-78 [13].
