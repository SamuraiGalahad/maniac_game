\section{ТЕХНИКО-ЭКОНОМИЧЕСКИЕ ПОКАЗАТЕЛИ}

\subsection{Ориентировочная экономическая эффективность}

В рамках данной работы расчёт экономической эффективности не предусмотрен.

\subsection{Предполагаемая потребность}

В видеоигровой индустрии существует постоянный спрос на новые хоррор-игры, как от известных международных компаний, так и от небольших объединений инди-разработчиков. Данный проект стремится создать жуткую внутриигровую локацию с множеством интерактивных предметов и компьютерным противником, чтобы предоставить страшное и интересное приключение игроку.

\subsection{Экономические преимущества разработки по сравнению с лучшими отечественными и зарубежными образцами или аналогами}

Данный проект является видеоигрой в жанре хоррор со следующими ключевыми особенностями:

\begin{enumerate}
    \item[6.3.1.] Ретро-стилизация игры;
    \item[6.3.2.] Продуманный внутриигровой (ИИ) противник;
    \item[6.3.3.] Подача сюжета игры через внутриигровые объекты: записки, журналы.
\end{enumerate}