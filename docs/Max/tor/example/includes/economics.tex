\section{ТЕХНИКО-ЭКОНОМИЧЕСКИЕ ПОКАЗАТЕЛИ}

\subsection{Ориентировочная экономическая эффективность}

В рамках данной работы расчёт экономической эффективности не предусмотрен.

\subsection{Предполагаемая потребность}

В видеоигровой индустрии существует постоянный спрос на новые хоррор-игры, как от известных международных компаний, так и от небольших объединений инди-разработчиков. Данный проект стремится создать жуткую внутриигровую локацию с множеством интерактивных предметов и компьютерным противником, чтобы предоставить страшное и интересное приключение игроку.

\subsection{Экономические преимущества разработки по сравнению с лучшими отечественными и зарубежными образцами или аналогами}

<<Универсальный парсер>> представляет собой индивидуальное решение, разработанное для предприятия <<СберАналитики>>, ключевые запрошенные заказчиком особенности:

\begin{itemize}
    \item настраиваемый кроулинг -- возможность переходить на ссылкам, найденным на изначально поданной странице
    \item упрощённая интеграция с остальными компонентами аналитической платформы
    \item независимость от первоначальной разметки страницы (универсальность)
    \item независимость от типа страницы (статичный HTML/динамичный SPA)
\end{itemize}