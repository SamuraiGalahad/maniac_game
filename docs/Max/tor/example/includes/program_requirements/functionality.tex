\subsection{Требования к функциональным характеристикам}

\subsubsection{Требования к составу выполняемых функций}

\begin{enumerate}
    % Сюжет - пока опускаем. В планах сделать "интро", в которой чел приезжает к дому, его учат управлению - это сюжет.
    % Карта
    \item[4.1.1.1.] Программа должна предоставлять игровую локацию, в которой будут находится\\
    игровой персонаж, ловушки, головоломки, интерактивные предметы и прочие объекты при начале игры;
    % Персонаж: появление
    \item[4.1.1.2.] Программа должна размещать игрового персонажа в определенном месте на игровой локации при начале игры;
    % Персонаж: стамина
    \item[4.1.1.3.] Программа должна реализовывать систему очков выносливости игрового персонажа,\\
    значение которых уменьшаются при определенных действиях игрока;
    \item[4.1.1.4.] Программа должна предоставлять возможность постепенного восстановления очков\\
    выносливости с определенным максимальным значением;
    % Персонаж: хп
    \item[4.1.1.5.] Программа должна реализовывать систему очков здоровья игрового персонажа с\\
    определенным максимальным значением;
    \item[4.1.1.6.] Программа должна предоставлять возможность постепенного восстановления очков\\
    здоровья (``регенерация'');
    \item[4.1.1.7.] Программа должна ``перезапускать'' игровую локацию и игрового персонажа при\\
    достижении значения очков здоровья нуля (``смерть''). При ``перезапуске'' игровая локация возвращается в изначальное состояние, значения очков здоровья и очков выносливости игрового персонажа восстанавливаются до определенных максимальных значений, игровой персонаж появляется в определенном месте на игровой локации;
    % Персонаж: передвижение
    \item[4.1.1.8.] Программа должна предоставлять пользователю возможность перемещения игрового\\
    персонажа (``шаг'');
    \item[4.1.1.9.] Программа должна предоставлять пользователю возможность быстрого перемещения\\
    игрового персонажа (``бег'') при достаточном значении очков выносливости;
    \item[4.1.1.10.] Программа должна уменьшать значение очков выносливости при быстром перемещении игрового персонажа;
    \item[4.1.1.11.] Программа должна предоставлять пользователю возможность прыгнуть игровому\\
    персонажу при достаточном значении очков выносливости;
    \item[4.1.1.12.] Программа должна уменьшать значение очков выносливости при прыжке игрового\\
    персонажа;
    % Персонаж: инвентарь
    \item[4.1.1.13.] Программа должна реализовывать систему хранения предметов игрового персонажа\\
    (``инвентарь'');
    \item[4.1.1.14.] Программа должна предоставлять возможность просмотра всех предметов в инвентаре игрового персонажа;
    \item[4.1.1.15.] Программа должна предоставлять возможность выбрать ``текущий предмет''\\
    (находящийся в руках) из инвентаря;
    \item[4.1.1.16.] Программа должна предоставлять возможность добавления подбираемого предмета,\\
    расположенного на игровой локации, в инвентарь;
    \item[4.1.1.17.] Программа должна предоставлять возможность удаления предмета из инвентаря.\\
    При этом, данный предмет появляется на игровой локации рядом с игровым персонажем;
    % Карта: фонарь
    \item[4.1.1.18.] Программа должна реализовывать ``фонарь'' в качестве подбираемого предмета,\\
    находящегося на игровой локации;
    \item[4.1.1.19.] Программа должна предоставлять возможность включения и отключения фонаря,\\
    если он является текущим предметом инвентаря;
    \item[4.1.1.20.] Программа должна освещать место игровой локации рядом с игровым персонажем\\
    при включенном фонаре;
    % Карта: аптечка
    \item[4.1.1.21.] Программа должна реализовывать ``аптечку'' в качестве подбираемого предмета,\\
    находящегося на игровой локации;
    \item[4.1.1.22.] Программа должна предоставлять возможность использовать аптечку, если она\\является текущим предметом инвентаря. При использовании аптечки игровой персонаж\\
    кратковременно получает ускоренную регенерацию, аптечка удаляется из инвентаря;
\end{enumerate}

\subsubsection{Организация входных данных}

Программа реализована как настольное приложение для операционной системы Windows. Данные от пользователя поступают в программу через клавиатуру и компьютерную мышь. \\
Иные требования к организации входных данных не предъявляются. 

\subsubsection{Организация выходных данных}

Программа выводит выходные данные пользователю на монитор компьютера и через выходные аудиоустройства.
