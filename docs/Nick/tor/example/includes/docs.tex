\section{ТРЕБОВАНИЯ К ПРОГРАММНОЙ ДОКУМЕНТАЦИИ}

\subsection{Состав программной документации}

В состав программной документации должны входить следующие компоненты:
\begin{enumerate}
    \item "3D Xоррор-игра от первого лица на Unreal Engine 5 для ПК". Техническое задание (ГОСТ 19.201-78) [7]
    \item "3D Xоррор-игра от первого лица на Unreal Engine 5 для ПК". Пояснительная записка (ГОСТ 19.404-79) [10]
    \item "3D Xоррор-игра от первого лица на Unreal Engine 5 для ПК". Руководство оператора (ГОСТ 19.505-79) [11];
    \item "3D Xоррор-игра от первого лица на Unreal Engine 5 для ПК". Текст программы (ГОСТ 19.401-78) [9];
    \item "3D Xоррор-игра от первого лица на Unreal Engine 5 для ПК". Программа и методика испытаний (ГОСТ 19.301-79) [8];
\end{enumerate}


\subsection{Специальные требования к программной документации}

Документы к программе должны быть выполнены в соответствии с ГОСТ 19.106-78 [6] и ГОСТами к каждому виду документа (см. п. 5.1.).
Пояснительная записка должна быть загружена в систему Антиплагиат через SmartLMS <<НИУ ВШЭ>>.
Техническое задание и пояснительная записка, титульные листы других документов должны быть подписаны руководителем разработки и исполнителем. Документация и программа сдается в электронном виде в формате pdf или docx в архиве формата zip или rar.

За три дня до защиты комиссии все материалы курсового проекта:
\begin{enumerate}
    \item программная документация
    \item программный проект
    \item исполняемый файл
    \item отзыв руководителя
    \item отчет системы Антиплагиат
\end{enumerate}
должны быть загружены одним или несколькими архивами в проект дисциплины <<Курсовой проект>> в личном кабинете в информационной образовательной среде SmartLMS <<НИУ ВШЭ>>.
