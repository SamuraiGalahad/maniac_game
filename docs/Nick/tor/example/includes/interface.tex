\subsection{Требования к интерфейсу}

Программа должна обладать интуитивным интерфейсом, не требующим от пользователя специальной подготовки. Программа стремится предоставить минималистичный интерфейс для глубокого погружения в игровой процесс.

\begin{enumerate}
    \item[4.3.1.] Интерфейс программы должен состоять из двух основных интерфейсов: интерфейс главного меню и интерфейс процесса игры;
    \item[4.3.2.] Интерфейс главного меню должен состоять из двух кнопок: ``Играть'', отвечающая за запуск игрового процесса и ``Выйти'', отвечающая за корректное закрытие программы;
    \item[4.3.3.] Интерфейс процесса игры должен отображать точку, которая показывает куда смотрит внутриигровой персонаж (``внутриигровой курсор'');
    \item[4.3.4.] Интерфейс процесса игры должен отображать значение очков выносливости внутриигрового персонажа (см. пункт 4.1.1.3.), значение очков здоровья внутриигрового персонажа (см. пункт 4.1.1.5.), текущий выбранный предмет инвентаря (если такой есть, см. пункт 4.1.1.15.);
    \item[4.3.5.] Значение очков здоровья внутриигрового персонажа должно отображаться в виде красных пятен, появляющихся на экране персонажа, в зависимости от текущего значения очков здоровья;
    \item[4.3.6.] Интерфейс процесса игры должен отображать название интерактивного предмета, если предмет расположен вблизи от внутриигрового персонажа и на такой предмет наведен внутриигровой курсор;
    \item[4.3.7.] Интерфейс процесса игры должен отображать сообщение о смерти внутриигрового персонажа при достижении значения очков здоровья нуля (см. пункт 4.1.1.7.). 
\end{enumerate}