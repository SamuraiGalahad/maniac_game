\subsection{Требования к функциональным характеристикам}

\subsubsection{Требования к составу выполняемых функций}

\begin{enumerate}
    % Сюжет - пока опускаем. В планах сделать "интро", в которой чел приезжает к дому, его учат управлению - это сюжет.
    % Меню
    \item[4.1.1.1.] Программа должна предоставлять возможность пользоватлелю загружать новую игровую\\
    сессию через стартовое меню;
    \item[4.1.1.2.] Программа должна предоставлять возможность пользователю посмотреть справку об игре\\
    из стартового меню;
    \item[4.1.1.3.] Программа должна предоставлять возможность выхода из игры через стартовое меню;
    \item[4.1.1.4.] Программа должна предоставлять возможность изменять громкость игры в меню \\настроек;
    \item[4.1.1.5.] Программа должна реализовывать возможность изменять экранный режим в меню \\настроек;
    \item[4.1.1.6.] Программа должна реализовать возможность изменять качество текстур в меню \\ настроек;
    \item[4.1.1.7.] Программа должна реализовать возможность выбирать сложность текущей игровой\\ 
    сессии в меню настроек;
    \item[4.1.1.8.] Программа должна предоставлять возможность менять яркость игрового освещения\\
    через меню настроек;
    \item[4.1.1.9.] Программа должна предоставлять возможность менять по отдельности\\
    громкость игрового окружения и громкость ИИ маньяка через меню настроек;
    % Меню ИИ
    \item[4.1.1.10.] Программа должна предоставлять пользователю возможность взаимодействовать с \\
    игровым миром через ИИ маньяка;
    \item[4.1.1.11.] Программа должна предоставлять пользователю возможность атаковать ИИ маньяка\\
    для использования эффекта изменения скорости перемещения маньяка \\по игровой локации;
    \item[4.1.1.12.] Программа должна реализовавывать атаку ИИ маньяка с потерей\\
    очков здоровья пользователя;
    \item[4.1.1.13.] Программа должна реализовывыть атаку ИИ маньяка с потерей\\ 
    очков выносливости пользователя;
    \item[4.1.1.14.] Программа должна реализовывать возможность перемещения\\
    ИИ маньяка ("шаг");
    \item[4.1.1.15.] Программа должна реализовывать возможность быстрого перемещения\\
    ИИ маньяка ("бег");
    \item[4.1.1.16.] Программа должна реализовывать возможность расставлять ловушки\\
    по игровой локации через ИИ маньяка;
    \item[4.1.1.17.] Программа должна реализовывать захвата камеры пользователя\\
    при использовании способности ИИ маньяка "похищение";
    \item[4.1.1.18.] Программа должна реализовывать анимацию ИИ маньяка;
    \item[4.1.1.19.] Программа должна реализовывать отображение информации\\
    о приближении ИИ маньяка к пользователю;
    % Карта: ловушки
    \item[4.1.1.20.] Программа должна реализовывать в окружении пользователя объекты\\
    ловушки, способные нанести урон игровому персонажу;
    \item[4.1.1.21.] Программа должна предоставвлять возможность обезвреживать ловушки\\
    по истечении времени с начала обезвреживания;
    \item[4.1.1.22.] Программа должна реализовывать анимацию атаки ловушки;
    \item[4.1.1.23.] Программа должна реализовывать анимацию обезвреживания ловушки;
    % Карта: Дом
    \item[4.1.1.24.] Программа должна реализовать локацию "Дом маньяка" с\\
    c возможностью перемещения игровых персонажей;
    \item[4.1.1.25.] Программа должна реализовывать анимацию внутреннего интерьера\\
    локации "Дом маньяка";
\end{enumerate}

\subsubsection{Организация входных данных}

Программа реагирует на ввод пользователя, который осуществляется через нажатия клавиш клавиатуры, левой кнопки мыши, передвижения мыши.
Эти входные данные включают в себя нажатия на кнопки мыши в процессе выбора опций в любых виджетах и меню.

\subsubsection{Организация выходных данных}

Программа визуализирует игровой процесс с использованием графических изображений. Она реагирует на изменения в отображении игрового мира в окне просмотра при перемещении игрового персонажа и вращении камеры. Кроме того, программа отображает различные виджеты и меню по командам, активируемым нажатием клавиш клавиатуры или левой кнопки мыши.
